%! Author = naimsg16
%! Date = 11/13/24

% Preamble
\documentclass[11pt]{article}

% Packages
\usepackage{amsmath}
\usepackage{amssymb}
\usepackage{wasysym}
\usepackage[makeroom]{cancel}
\setlength{\parindent}{0pt}
% Document
\begin{document}
\textbf{Demostra per inducció la següent propietat dels arbres binaris no buits: el
nombre de fulles ($N_0$) d’un arbre binari no buit és exactament igual al nombre
de nodes de grau 2 ($N_2$) més 1.}
\begingroup
    \center{Hipòtesi d'inducció: $N_0 = N_2 + 1$}\\[0.25cm]
\endgroup
\\
\textbf{Cas trivial:} $n = 1$\\
Si té un sol node (arrel), llavors té 0 nodes de grau 2 $N_2 = 0$ y un node de grau 0 $N_0 = 1$, per tant:
\[1 = 0 + 1 \Longrightarrow 1 = 1 \quad\checkmark\]

\textbf{Cas inductiu:} En un arbre binari, només podem afegir un node de dues formes: a un node de grau 0 (sense fills) o a un node de grau 1 (amb un fill).\\
    \begin{itemize}
        \item\textbf{Afegir un node a un de grau 0:}\\
            En fer això creem un nou node de grau 0 (incrementem $N_0$) però a la vegada el seu node pare deixa de ser un node de grau 0 (disminuïm $N_0$).
            \begin{gather*}
                N_0+1-1 = N_2 + 1\\
                N_0 = N_2 +1 \quad \checkmark\\
            \end{gather*}
        \item \textbf{Afegir un node a un de grau 1:}\\
             En fer això creem un nou node de grau 0 (incrementem $N_0$) però transformem el node de grau 1 en un de grau 2 (incrementem $N_2$).
            \begin{gather*}
                N_0+1 = N_2 + 1 + 1\\
                N_0 +\cancel{1} = N_2 + 1 +\cancel{ 1}\\
                N_0 = N_2 +1 \quad \checkmark\\
            \end{gather*}
    \end{itemize}

\end{document}